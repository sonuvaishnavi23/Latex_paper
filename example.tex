\documentclass{article}
\usepackage{amsmath, amssymb}
\title{\underline{\textbf{MATHEMATICS}}}
\date{}

\begin{document}
\maketitle
\begin{enumerate}
    
\item Q1 Suppose $a, b$ denote the distinct real roots of the quadratic polynomial $x^2+20x-2020$ and suppose $c, d$ denote the distinct complex roots of the quadratic polynomial $x^2+20x+2020$. Then the value of
\[
    ac(a-c) + ad(a-d) + bc(b-c) + bd(b-d)
\]
    is:
    
    \begin{enumerate}
        \item  0
        \item  8000
        \item  80800
        \item  16000
    \end{enumerate}

\item Q2 If the function $f: \mathbb{R} \to \mathbb{R}$ is defined by $f(x)=|x|(x-\sin x)$, then which of the following statements is TRUE?

    \begin{enumerate}
        \item  $f$ is one-one, but NOT onto
        \item  $f$ is onto, but NOT one-one
        \item  $f$ is BOTH one-one and onto
        \item  $f$ is NEITHER one-one NOR onto
    \end{enumerate}

\item Q3 Let the functions $f: \mathbb{R} \to \mathbb{R}$ and $g: \mathbb{R} \to \mathbb{R}$ be defined by 

$f(x) = e^{(x - 1)} - e^{- |x - 1|}$ and $ g(x) = \frac{1}{2} \left(e^{(x - 1)} + e^{(1 - x)}\right)$

    Then the area of the region in the first quadrant bounded by the curves $y = f(x)$, $y = g(x)$ and $x = 0$ is:

    \begin{enumerate}
        \item  $(2 - \sqrt{3}) + \frac{1}{2} (e - e^{-1})$
        \item  $(2 + \sqrt{3}) + \frac{1}{2} (e - e^{-1})$
        \item  $(2 - \sqrt{3}) + \frac{1}{2} (e + e^{-1})$
        \item  $(2 + \sqrt{3}) + \frac{1}{2} (e + e^{-1})$
    \end{enumerate}

\item Q4 Let $a, b$ and $\lambda$ be positive real numbers. Suppose $P$ is end point of the latus rectum of the parabola $y^2 = 4\lambda x$, and the suppose ellipse $\frac{x^2}{a^2} + \frac{y^2}{b^2} = 1$ passes through the point $P$, If the tangents to the parabola and the ellipse at the point $P$ are perpendicular to each other, then the eccentricity of the ellipse is:

    \begin{enumerate}
        \item  $\frac{1}{\sqrt{2}}$
        \item  $\frac{1}{2}$
        \item  $\frac{1}{3}$
        \item  $\frac{2}{5}$
    \end{enumerate}

\item Q5 Let $C_1$ and $C_2$ be two biased coins such that the probabilities of getting heads in a single toss are $\frac{2}{3}$ and $\frac{1}{3}$, respectively. Suppose $\alpha$ is the number of heads that appear when $C_1$ is tossed twice, independently, and suppose $\beta$ is the number of heads that appear when $C_2$ is tossed twice, independently. Then the probability that the roots of the quadratic polynomial $x^2 - \alpha x +\beta$ are real and equal, is:

    \begin{enumerate}
        \item  $\frac{40}{81}$
        \item  $\frac{20}{81}$
        \item  $\frac{1}{2}$
        \item  $\frac{1}{4}$
    \end{enumerate}

\item Q6 Consider all rectangles lying in the region:

\[
	\{(x,y) \in \mathbb{R} \times \mathbb{R} : 0 \leq x \leq \frac{2\pi} 1, 0 \leq y \leq \frac{2\sin(2x)}{2}\}
\]

    and having one side on the x-axis. The area of the rectangle which has the maximum perimeter among all such rectangles, is:

    \begin{enumerate}
        \item  $\frac{3\pi}{2}$
        \item  $\pi$
        \item  $\frac{\pi}{2\sqrt{3}}$
        \item  $\frac{\pi\sqrt{3}}{2}$
    \end{enumerate}

\item Q7 Let the function $f: \mathbb{R} \to \mathbb{R}$ be defined by $f(x) = x^3 - x^2 + (x - 1) \sin x$ and let $g: \mathbb{R} \to \mathbb{R}$ be an arbitrary function. Let $fg: \mathbb{R} \to \mathbb{R}$ be the product function defined by $(fg)(x) = f(x)g(x)$. Then which of the following statements is/are TRUE?

    \begin{enumerate}
        \item  If $g$ is continuous at $x = 1$ then $fg$ is differentiable at $x = 1$
        \item  If $fg$ is differentiable at $x = 1$ then $g$ is continuous at $x = 1$
        \item  If $g$ is differentiable at $x = 1$ then $fg$ is differentiable at $x = 1$
        \item  If $fg$ is differentiable at $x = 1$ then $g$ is differentiable at $x = 1$
    \end{enumerate}

\item Q8 Let $M$ be a $3 \times 3$ invertible matrix with real entries and let $I$ denote the $3 \times 3$ identity matrix. If $ M^{-1} = \text{adj}(\text{adj } M)$ then which of the following statements is/are ALWAYS TRUE?

    \begin{enumerate}
        \item  $M = I$
        \item  $\det M = 1$
        \item  $M^2 = I$
        \item  $(\text{adj } M)^2 = I$
    \end{enumerate}
\item Q9 Let $S$ be the set of all complex numbers $z$ satisfying $|z^2 + z + 1| = 1$. Then which of the following statements is/are TRUE?

    \begin{enumerate}
        \item  $|z + \frac{1}{2}| \leq \frac{1}{2}$ for all $z \in S$
        \item  $|z| \leq 2$ for all $z \in S$
        \item  $|z + \frac{1}{2}| \geq \frac{1}{2}$ for all $z \in S$
        \item  The set $S$ has exactly four elements
    \end{enumerate}

\item Q10 Let $x, y$ and $z$ be positive real numbers. Suppose $x, y$ and $z$ are the lengths of the sides of a triangle opposite to its angles $X, Y$ and $Z$, respectively. If 

\[
    \tan\left(\frac{X}{2}\right) + \tan\left(\frac{Z}{2}\right) = \frac{2y}{x + y + z}
\]

    then which of the following statements is/are TRUE?

    \begin{enumerate}
        \item  $2Y = X + Z$
        \item  $Y = X + 2$
        \item  $\tan\left(\frac{X}{2}\right) = \frac{x}{y + x}$
        \item  $x^2 + z^2 - y^2 = xz$
    \end{enumerate}

\item Q11 Let $L_1$ and $L_2$ be the following straight lines:

\[
    L_1 :\frac{x - 1}{1} = \frac{y}{-1} = \frac{z - 1}{3}, \quad 
    L_2 : \frac{x - 1}{-3} = \frac{y}{-1} = \frac{z - 1}{1}
\]

    Suppose the straight line 

\[
    L : \frac{x - a}{l} = \frac{y - 1}{m} = \frac{z - y}{-2}
\]

    lies in the plane containing $L_1$ and $L_2$ and passes through the point of intersection of $L_1$ and $L_2$. If the line $L$ bisects the acute angle between the lines $L_1$ and $L_2$, then which of the following statements is/are TRUE?

    \begin{enumerate}
        \item  $\alpha - \gamma = 3$
        \item  $l + m = 2$
        \item  $\alpha - \gamma = 1$
        \item  $l + m = 0$
    \end{enumerate}

\item Q12 Which of the following inequalities is/are TRUE?

    \begin{enumerate}
        \item  $\int_0^1 x \cos x \,dx \geq \frac{3}{8}$
        \item  $\int_0^1 x \sin x \,dx \geq \frac{3}{10}$
        \item  $\int_0^1 x^2 \cos x \,dx \geq \frac{1}{2}$
        \item  $\int_0^1 x^2 \sin x \,dx \geq \frac{2}{9}$
    \end{enumerate}

\item Q13 Let $m$ be the minimum possible value of $\log_3(3^{y_1} + 3^{y_2} + 3^{y_3})$ $y_1, y_2, y_3$ are real numbers for which $y_1 + y_2 + y_3 = 9$. Let $M$ be the maximum possible value of $\log_3(x_1) + \log_3(x_2) + \log_3(x_3)$ where $x_1, x_2, x_3$ are positive real numbers for which $x_1 + x_2 + x_3 = 9$. Then the value of $\log_2(m^3) + \log_3(M^2)$ is \underline{\hspace{2cm}}.

\item Q14 Let $a_1, a_2, a_3, \dots $ be a sequence of positive integers in arithmetic progression with common difference 2. Also, let $b_1, b_2, b_3, \dots $ be a sequence of positive integers in geometric progression with common ratio 2. If $a_1 = b_1 = c$, then the number of all possible values of $c$ for which the equality 

\[
    2(a_1 + a_2 + \dots + a_n) = b_1 + b_2 + \dots + b_n
\]

    holds for some positive integer $n$, is \underline{\hspace{2cm}}.

\item Q15 Let $f: [0,2] \to \mathbb{R}$ be the function defined by

\[
    f(x) = (3 - \sin(2\pi x)) \sin\left(\pi x - \frac{\pi}{4}\right) - \sin(3\pi x + \frac{\pi}{4}).
\]
If $\alpha, \beta \in [0,2]$ are such that$\{ x \in [0,2] : f(x) \geq 0 \} = [\alpha, \beta],$then the value of $\beta - \alpha$ is \underline{\hspace{2cm}}.

\item Q16 In a triangle $PQR$, let 

   \[\vec{a} = \overrightarrow{QR}, \quad \vec{b} = \overrightarrow{RP}, \quad \vec{c} = \overrightarrow{PQ}.\]

If 
    \[|\vec{a}| = 3,  |\vec{b}| = 4,]and [\frac{\vec{a} \cdot (\vec{c} - \vec{b})}{\vec{c}\cdot(\vec{a} - \vec{b})}=][\frac{|\vec{a}|}{|\vec{a}| + |\vec{b}|}]\]

    then the value of $|\vec{a} \times \vec{b}|^2$ is \underline{\hspace{2cm}}.

\item Q17 For a polynomial $g(x)$ with real coefficients, let $m_g$ denote the number of distinct real roots of $g(x)$. Suppose $S$ is the set of polynomials with real coefficients defined by\[S = \{(x^2 - 1)^2 (a_0 + a_1 x + a_2 x^2 + a_3 x^3) : a_0, a_1, a_2, a_3 \in \mathbb{R} \}.\]For a polynomial $f'$, let $f'$ and $f''$ denote itsfirst and second order derivatives, respectively. Then the minimum possible value of $(m_{f'} + m_{f''})$ where $f \in S$, is \underline{\hspace{2cm}}.

\item Q18 Let $e$ denote the base of the natural logarithm. The value of the real number $a$ for which the right-hand limit

\[
    \lim_{x \to 0^+} \frac{(1 - x)^{\frac{1}{x}} - e^{-1}}{x^a}
\]

    is equal to a nonzero real number, is \underline{\hspace{2cm}}.


\end{enumerate}

\end{document}

